\section{Saving and Investment in the Open Economy}

\subsection{Balance of Payments Accounting}

\begin{definition}
    (Current Account) The \textbf{current account} measures a country's trade in currently produced goods and services, along with unilateral transfers between countries. 
    \[
        CA = NX + NFP + NUT
    \]
    Where 
    \begin{itemize}
        \item $NX$: net exports of goods and services
        \item $NFP$: net income from abroad (primary income), approximated by $NFP$
        \item $NUT$: net unilateral transfers (secondary income)
    \end{itemize} 
\end{definition}

\begin{remark}
    $CA$ is equal to the amount of funds that a country has available for net foreign lending
\end{remark}


\begin{definition}
    (Capital and financial account): The \textbf{capital and financial account} consists of 
    \begin{itemize}
        \item capital account: unilateral transfers of assets 
        \item financial account: transactions involving flow of assets
    \end{itemize} 
\end{definition}

\begin{definition}
    (Official settlements balance): \textbf{Official reserve assets} are assets, other than domestic money or securities, held by central banks and which can be used in making international payments. \\

    The \textbf{official settlements balance}, or \textbf{balance of payments}, is the net increase in a country's official reserve assets.
\end{definition}

\begin{result}
    In each period, the current account balance and the capital and financial account balance must sum to zero. 
    \[
        CA + KFA = 0
    \]

    Every international transaction involves a swap of goods and services for assets between countries. The two sides of the swap have offsetting effects on $CA + KFA$
\end{result}


\begin{example}
    The following are equivalent 
    \begin{itemize}
        \item $CA$ surplus of \$10M
        \item $KFA$ deficit of \$10M
        \item net \textbf{acquisition} of foreign assets of \$10M
        \item net foreign lending of \$10M
        \item net exports of \$10M, assuming $NFP = NUT = 0$
    \end{itemize} 
\end{example}

\subsection{Goods market equilibrium in an open economy}
\begin{theorem}
    (National income accounting identity) Assuming $NUT = 0$
    \[
        S = I + CA = I + (NX + NFP)
    \]
    i.e. national saving ($S$) can be used to 
    \begin{enumerate}
        \item increase capital stock through $I$
        \item increase stock of net foreign assets by \textbf{lending to foreigners}
    \end{enumerate}
\end{theorem}

\begin{theorem}
    (Goods market equilibrium for an open economy) At goods market equilibrium, actual national saving and investment match their desired levels.

    i.e. The desired amount of national saving is equal to the desired amount of domestic investment plus the amount lent abroad
    \[
        S^d = I^d  + CA = I^d + (NX + NFP)
    \]

    For simplicity, assuming $NFP = 0$, the goods market equilibrium condition is 
    \[
        S^d = I^d + NX
    \]
    This is equivalent to 
    \[
        Y = C^d + I^d +G + NX
    \]
    and
    \[
    NX = Y - (C^d + I^d + G)
    \]

    The last equation can be interpreted as: at goods market equilibrium, the amount of net exports equals total output less desired absoprtion (spending by domestic residents). 
    \begin{itemize}
        \item output $>$ absorption $\implies$ $NX > 0$
        \item output $<$ absorption $\implies$ $NX < 0$
    \end{itemize} 
\end{theorem}

\subsection{Small Open Economy}

\begin{definition}
    (Small open economy): A \textbf{small open economy} is defined as an economy too small to affect world real interest rate.
\end{definition}

\begin{remark}
    In a small open economy, the interest rate is fixed at $r^w$. Changes to $S$ and $I$ changes $CA$.

\begin{center}
\includegraphics[width = 0.5\textwidth]{graphs/fig_5_1.jpg}
\end{center}
\end{remark}

\begin{example}
    Change in $G$ \\ 

    \begin{itemize}
        \item $G \uparrow \implies S = (Y - C - G\uparrow) \downarrow \implies NX \downarrow = CA \downarrow$ 
    \end{itemize} 
\begin{center}
\includegraphics[width = 0.5\textwidth]{graphs/fig_5_2.jpg}
\end{center}
\end{example}

\begin{example}
    Temporary adverse supply shock in small open economy \\ 
\begin{center}
\includegraphics[width = 0.5\textwidth]{graphs/fig_5_3.jpg}
\end{center}
\end{example}

\begin{example}
    Increase in $MPK^f$ \\
    \begin{itemize}
        \item $I \uparrow \implies CA \downarrow$
    \end{itemize} 
\begin{center}
\includegraphics[width = 0.5\textwidth]{graphs/fig_5_4.jpg}
\end{center}
\end{example}

\subsection{Large Open Economy}
\begin{definition}
    (Large open economy): A \textbf{large open economy} affects world interest rates. 

    The large open economy model is analogous to a two-economy model consisting of a domestic and a world economy.
\end{definition}

\begin{remark}
   In a large open economy / two economy model, the world interest rate is the interest rate such that desired international lending by one economy matches desired international borrowing by the other economy. \\

   We know that world supply of goods matches world demand for goods 
   \[
       Y + Y_{For} = C^d + I^d + G + C^d_{For} + I^d_{For} + G_{For}
   \]

   Hence, world saving matches world investment 
   \[
       \underbrace{(Y - G^d - G)}_{S^d} + \underbrace{(Y_{For} - C^d_{For} - G_{For})}_{S^d_{For}} = I^d + I^d_{For}
   \]
   Hence, desired international borrowing matches desired international lending 
   \[
       \underbrace{(S^d - I^d)}_{CA} + 
       \underbrace{(S^d_{For} - I^d_{For})}_{CA_{For}} = 0
   \]
   
   
\begin{center}
\includegraphics[width = 0.8\textwidth]{graphs/fig_5_5.jpg}
\end{center}
   

\end{remark}

\begin{example}
    Increase in government spending
    \begin{itemize}
        \item $G^H \uparrow \implies S^H = ( \bar{Y}^H - C^H - G^{H} \uparrow ) \downarrow $ 
        \item domestic savings curve shifts left
        \item world savings curve shifts left 
        \begin{itemize}
            \item world savings shifts by the manitude such that $r^W$ brings $NX^H$ and $NX^F$ into equilibrium, i.e. 
            \[
                NX^H = NX^F \iff CA^H = CA^F
            \]
        \end{itemize} 
        \item composition of GDP: $\underset{no\ change}{\bar{Y}^{H}} = \underset{a\ little}{C \shortdownarrow} + I^H \downarrow + G^H \uparrow + NX^H \downarrow$
    \end{itemize} 
\begin{center}
\includegraphics[width = 0.8\textwidth]{graphs/fig_5_6.jpg}
\end{center}
\end{example}

\subsection{Fiscal policy and Current Account}
\begin{proposition}
    An increase in government budget deficit will raise the current account deficit only if the increase in budget deficit reduces national saving.
\end{proposition}



\newpage