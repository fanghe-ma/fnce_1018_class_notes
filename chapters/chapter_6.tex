\section{Long Run Economic Growth}

\subsection{Growth Accounting}

\begin{definition}
    (Capital-labor ratio) The capital-labor ratio is the amount of capital stock per worker, denoted 
    \[
        k_t = \frac{K_t}{N_t}
    \]
\end{definition}

\begin{definition}
    (Solow Growth Steady State) The Solow steady state is a situation in which the economy's output per worker, consumption per worker, and capital stock per worker are constant. 
\end{definition}

\begin{result}
    (Growth accounting equation) From the production function
    \[
        Y = AF(K, N)
    \]
    The relationship between output, input and productivity growth is 
    \[
        \frac{\triangle Y}{Y} = \frac{\triangle A}{A} + a_K \frac{\triangle K}{K} + a_N \frac{\triangle N}{N}
    \]
    Where 
    \begin{itemize}
        \item $a_K = $ elasticity of output wrt capital
        \item $a_N = $ elasticity of output wrt labor
    \end{itemize} 
\end{result}

\subsection{Solow Growth Model}

\begin{result}
    (Solow Growth Model) \\

    Denote 
    \begin{itemize}
        \item $N_t$: the population in year $t$ $N_t$
        \item $Y_t$: output in year $t$
        \item $I_t$: gross investment in year $t$
        \item $C_t$: consumption in year $t$
        \item $S_t$: saving in year $t$
        \item $y_t = \frac{Y_t}{N_t}$: output per worker in year $t$
        \item $c_t = \frac{C_t}{N_t}$: consumption per worker in year $t$
        \item $k_t = \frac{K_t}{N_t}$: capital stock per worker in year $t$
    \end{itemize} 

    Assume
    \begin{itemize}
        \item $N_t$ grows at fixed rate $n$
        \item economy is closed and $G = 0$, which implies output is either consumed or invested to grow capital stock, then 
        \[
            C_t = Y_t - I_t
        \]
        \item $S_t$ is proportional to current income for some fixed saving rate 
        \[
            S_t = s Y_t
        \]
    \end{itemize} 

    The per-worker production function is 
    \[
    c_t = Af(k_t) - (n + d) k_t
    \]
    

    At steady state, capital grows at $n$, hence 
    \[
        I_t = (n +d) K_t
    \]

    Steady state consumption is therefore 
    \[
        C_t = Y_t - (n + d) K_t
    \]

    In per worker terms, 
    \[
    c_t = Af(k_t) - (n + d) k_t
    \]

    Since national savings equal investment 
    \[
        S_t = sY_t = (n + d) K_t \implies f A f(k_t) = (n + d) k_t
    \]

    At equilibrium, 

    \begin{center}
    \includegraphics[width = 0.5\textwidth]{graphs/fig_6_3.jpg}
    \end{center}
\end{result}

\begin{definition}
    (Golden Rule capital labor ratio): The golden rule capital labor ratio is the capital-labor ratio that maximizes consumption. 

    \begin{center}
    \includegraphics[width = 0.5\textwidth]{graphs/fig_6_2.jpg}
    \end{center}
\end{definition}
\begin{remark}
    There is no equilibrating mechanism to bring $k$ to $k^{**}$.
\end{remark}

\begin{remark}
    Effects of change in $s$, saving rate
    \begin{center}
    \includegraphics[width = 0.5\textwidth]{graphs/fig_6_4.jpg}
    \end{center}
\end{remark}

\begin{remark}
    Effects of change in $n$, population growth 
    \begin{center}
    \includegraphics[width = 0.5\textwidth]{graphs/fig_6_5.jpg}
    \end{center}
\end{remark}

\begin{remark}
    Effects of change in $A$, productivity growth
    \begin{center}
    \includegraphics[width = 0.5\textwidth]{graphs/fig_6_6.jpg}
    \end{center}
\end{remark}

\subsection{Endogenous Growth Theory}
\begin{remark}
    According to Solow Growth Model, sustained growth in output per capital can only be achieved through sustained productivity growth, which is exogenous. \\

    Endogenous growth theory allows for endogenous productivity growth.
\end{remark}

\begin{example}
    One endogenous model is 
    \[
        Y = AK
    \]
    Note that in this model, the marginal product of capital is $A$, and the marginal product of capital does not depend on $K$ (no diminishing marginal products). \\

    Assume that national saving is a constant fraction of output 
    \[
        S = sY = sAK
    \]
    Also, 
    \[
    I = \triangle K +  dK
    \]
    Hence 
    \begin{align*}
        & I = S \\
        \implies & \triangle K = dK = sAK \\
        \implies & \frac{\triangle K}{K} = sA - d \\
        \implies & \frac{\triangle Y}{Y} = sA - d
    \end{align*}
    
    
\end{example}































\newpage