\section{Measurement and Structure of the National Economy}

\subsection{National Income Accounting: Production, Income, Expenditure}

\textbf{National income acounting} states that ecept for incomplete or misreported data, \textit{the following three approaches give identical measurements of the amount of current economic activity}
\begin{itemize}
    \item Product approach: the amount of final output produced 
    \item Income approach: the incomes received by produces of output
    \item Expenditure approach: the amount of spending by final purchasers of output
\end{itemize} 

\subsection{Gross Domestic Product}

\begin{definition}
    \textbf{Gross Domestic Product (Product Approach)} is defined as the market value of final goods and services newly produced within a nation during a fixed period of time.
\end{definition}

\begin{remarks}
    \textbf{Market value}: prices at which goods and services are sold
    \begin{itemize}
        \item (+): Allows adding of production of different goods and services
        \item (-): Some goods or services are not sold in formal markets, denoted \textbf{nonmarket goods and services}.
    \end{itemize}
\end{remarks}

\begin{remarks}
    \textbf{Newly produced}: GDP only includes goods and services produced within current period.
\end{remarks}

\begin{remarks}
    \textbf{Final}: Goods and services are final if they are not intermediate. Goods and services are intermediate if they are used up in the production of others \textbf{in the same period they were produced}. By this definition, \textbf{inventories} are a final good. 
\end{remarks}

\begin{definition}
    \textbf{Gross National Product} is defined as the market value of final goods and services newly produced \textit{by domestic factors of production} during the current period.
\end{definition}

\begin{definition}
    \textbf{NFP} or \textbf{Net factor payments from abroad} is defined as the income paid to domestic factors of production by the rest of the world minus income paid to foreign factors of production by the domestic economy.
    \[
        \text{GDP} = \text{GNP} - \text{NFP}
    \]
\end{definition}

\begin{definition}
    \textbf{Income-Expenditure Identity}: The \textbf{Gross Domestic Product} (Expenditure Approach) is defined as the total spending on final goods and services produced within a nation during a specified period of time. 
    \[
        Y = C + I + G + NX
    \]
    Where 
    \begin{align*}
        Y = \text{GDP} &= \text{ total output / production} \\
        &= \text{ total income} \\
        &= \text{ total expenditure} \\
        C &= \text{ consumption} \\
        I &= \text{ investment} \\
        G &= \text{ government purchases of goods and services} \\
        NX &= \text{ net exports of goods and services}
    \end{align*}
\end{definition}

\begin{remarks}
    \textbf{Consumption} is spending by households on final goods and services, \textbf{including those produced abroad}. It comprises of
    \begin{itemize}
        \item consumer durables
        \item nondurables
        \item services
    \end{itemize} 
\end{remarks}

\begin{remarks}
    \textbf{Investment} includes spending on
    \begin{itemize}
        \item fixed investment: spending on new capital goods
        \begin{itemize}
            \item business fixed investment
            \item residential investment
        \end{itemize} 
        \item inventory investment: spending on inventory holdings
    \end{itemize} 

    Note that goods that are \textbf{produced} but \textbf{unsold} are considered investments in inventory. \\

    Note also that investment includes spending on foreign-produced goods.
\end{remarks}

\begin{remarks}
    \textbf{Government purchases} of goods and services include expenditure by government for a currently produced good or service, foreign or domestic. \\

    Note that $G$ excludes
    \begin{itemize}
        \item transfers 
        \item interest payment on national debt
    \end{itemize} 

    Government spending on fixed investments or inventory is counted under $G$, not $I$.
\end{remarks}

\begin{remarks}
    \textbf{Net Exports} are exports minus imports.
    \begin{itemize}
        \item Exports: goods and services produced within a country purchased by foreigners, \textbf{added to total spending} because they represent spending on final goods and services within the country 
        \item Imports: vice versa
    \end{itemize} 
\end{remarks}

\begin{definition}
    \textbf{National Income} is defined as the sum of 
    \begin{itemize}
        \item Compensation of employees
        \item Proprietors' income 
        \item Rental income of persons
        \item Corporate profits
        \item Net interest
        \item Taxes on production and imports
        \item Business current trasnfer payments
        \item Current surplus of government enterprises
    \end{itemize} 

    \textbf{GDP} (Income Approach) is defined as National Income plus
    \begin{itemize}
        \item statistical discrepancy
        \item depreciation: consumption of fixed capital
    \end{itemize} 
\end{definition}

\begin{remarks}
\[
\underbrace{
\underbrace{
    \text{National Income}
    + \text{Statistical Discrepancy}
}_{\text{Net National Product}}
+ \text{Depreciation}
}_{\text{Gross Domestic Product}}
+ \text{NFP}
= \text{Gross National Product}
\]
\end{remarks}

\begin{definition}
    \textbf{Private disposable income} is the amount of income that the private sector has available to spend.
    \[
        \text{private disposable income} = Y + NFP + TR + INT - T
    \]
    where 
    \begin{align*}
        Y &= \text{GDP} \\
        NFP &= \text{net factor payment from abroad} \\
        TR &= \text{transfers received from the government} \\
        T &= \text{taxes}
    \end{align*}
\end{definition}

\begin{definition}
    \textbf{Net government income} is taxes paid by private sector, minus payments from government to the private sector 
    \[
        \text{net government income} = T - TR - INT
    \]
\end{definition}

\subsection{Savings and Wealth}

\textbf{Saving} is the excess of current income over current needs. 

\begin{definition}
    \textbf{Private saving} is defined as the difference of private disposable income and consumption 
    \begin{align*}
        S_{pvt} &= \text{private disposable income} - \text{ consumption} \\
        &= (Y + NFP - T + TR + INT) - C
    \end{align*}
\end{definition}

\begin{remarks}
    Note that \textbf{investments} is not subtracted from private disposable income even though it constitutes private spending, because it \textit{does not satisfy current needs}.
\end{remarks}

\begin{definition}
    \textbf{Government saving} is defined as net government income, less goernment purchases of goods and services 
    \begin{align*}
        S_{govt} &= \text{ net government income } - \text{government purchases} \\
        &= (T - TR - INT) - G
    \end{align*}
\end{definition}

\begin{remark}
    Government purchase is technically divided into government consumption ( to meet current needs) and government investment (on long-lived assets). For most purposes, including the above definition, $G$ is assumed to be entirely made up of government consumption.
\end{remark}

\begin{remark}
    \textbf{Government budget surplus} is defined as government receipts less outlays 
    \begin{align*}
        \text{budget surplus} &= \text{government receipts} - \text{government outlays} \\
        &= T - (G + TR + INT) \\
        &= S_{govt}
    \end{align*}
\end{remark}

\begin{definition}
    \textbf{National saving} is defined as the sum of private saving and government saving 
    \begin{align*}
        S &= S_{pvt} + S_{govt} \\
        &= (Y + NFP - T + TR + INT - C) + ( T - TR - INT - G) \\
        &= Y + NFP - C - G
    \end{align*}
\end{definition}

\begin{definition}
    \textbf{Current Account Balance} is defined as payments received from abroad in exchance for currently produced goods and services, minus payments made to foreigners by the domestic economy for currently produced goods and services and net unilateral transfers.
    \[
    CA = NX + NFP + NUT
    \]
\end{definition}

\begin{remark}
    $CA, NX, NFP, NUT$ all follow the money.
    \begin{itemize}
        \item money coming in: $CA > 0$
        \item money leaving: $CA < 0$
    \end{itemize} 
\end{remark}

\begin{definition}The \textbf{Financial Account} balance is defined as the additive inverse of the current account. 
    \[
        FA = - CA
    \] 
    on the assumption that any domestic currency spent on foreign goods in excess of what foreigners need to purchase domestic goods will be spent purchasing domestic financial assets.
\end{definition}
\begin{remark}
    A company is a \textbf{net creditor} (i.e. lending abroad) if 
    \[
        FA < 0 \iff CA > 0 \iff NX > 0 \text{ assuming } NFP = NUT = 0
    \]
\end{remark}


\begin{theorem}
    \textbf{Uses-of-saving identity} The economy's private saving is used in 3 ways 
    \[
        S_{pvt} = I + (- S_{govt}) + CA
    \]
    \begin{itemize}
        \item Investment: firms borrow from savers to finance investment
        \item Government budget deficit: government finances deficits by borrowing from private savers
        \item Current account balance: foreigners borrow from US private savers to fund payments to the US
    \end{itemize} 
\end{theorem}

\begin{proof}
    We begin by substituting $Y = C + I + G + NX$ in the definition of national saving 
    \begin{align*}
        S &= Y + NFP - C - G \text{ by definition} \\
        &= (C + I + G + NX) + NFP - C - G \text{ by substituting } Y \\
        &= I + (NX + NFP) \\
        &= I + (CA) \text{ by definition} \\
        S_{pvt} + S_{govt} &= I + CA  \\
        S_{pvt} &= I + (- S_{govt}) + CA
    \end{align*}
\end{proof}

\begin{definition}
    \textbf{National Wealth} is the total wealth of residents in a country, consisting of 
    \begin{itemize}
        \item domestic physical assets, i.e. capital goods and land
        \item net foreign assets, comprising of 
        \begin{itemize}
            \item (+) foreign physical and financial assets 
            \item (-) foreign physical and financial liabilities 
        \end{itemize} 
    \end{itemize} 
\end{definition}

\begin{remark}
    Domestic financial assets held by domestic residents are \textbf{not} part of national wealth because each domestic financial asset is matched by a domestic financial liability.
\end{remark}

\begin{remark}
    National wealth can change through
    \begin{itemize}
        \item Change in value of existing assets and liabilities
        \item National saving
        \begin{itemize}
            \item Since $S = I + CA$, every dollar in savings increases domestic capital stock or domestic net foreign assets. \textit{Increases in national saving increase national wealth dollar for dollar}.
        \end{itemize} 
    \end{itemize} 
\end{remark}

\subsection{Real GDP, Price Indexes, Inflation}

\begin{definition}
    A \textbf{real} variable is one that measures the physical quantity of economic activty using the prices of a base year. 
\end{definition}

\begin{definition}
    A \textbf{price index} is a measure of the average level of prices for some specified set of goods and services, relative to the prices in a specified base year. 
\end{definition}

\begin{definition}
    The \textbf{GDP deflator} is a price index that measures the overall level of prices of goods and services included in GDP. 
    \[
        \text{GDP deflator} = 100 \times \frac{ \text{nominal GDP}}{ \text{real GDP}}
    \]
\end{definition}

\begin{definition}
    The \textbf{Consumer Price Index}, or \textbf{CPI}, measures the prices of consumer goods, calculated as 100 times the current cost of a specific baset of consumer items divided by the cost of the same basket in the reference base period.
\end{definition}

\begin{remark}
    The CPI tends to overstate increases in cost of living because of
    \begin{itemize}
        \item improvements in quality of goods and services
        \item substition bias
        \begin{itemize}
            \item CPI doesn't account for substitution away from the specified basket of goods
        \end{itemize} 
    \end{itemize} 
\end{remark}

\begin{definition}
    \textbf{Inflation rate} for a given period is defined as the percentage change in the price index in the same period.
    \[
        \pi_{t + 1} = \frac{P_{t + 1} - P_t}{P_t} = \frac{\triangle P_{t+1}}{P_t}
    \]
\end{definition}

\begin{remark}
    The FED's preferred inflation measure is the \textbf{Personal Consumption Expenditures (PCE)} price index, which measures consumer prices in the national income and product accounts.
    \begin{itemize}
        \item Headline inflation: overall change in PCE price Index
        \item Core inflation: change in PCE excluding \textbf{food and energy}
    \end{itemize} 

    Differences between PCE and PCI 
    \begin{itemize}
        \item PCE is based on actual hosuehold expenditure, avoids substitution bias
        \item PCE is broader than CPI
        \item PCE can be revised when better data is available
        \item PCE is chain-weighted
        \item CPI is based on average spending habits of \textbf{urban population}
    \end{itemize} 
\end{remark}

\subsection{Interest rates}

In general, interest rate is a rate of return promised by a borrower to a lender.

\begin{definition}
    The \textbf{real interest rate} on an asset is the rate at which the real value or purchasing power of the asset increases over time. \\


    The \textbf{nominal interest rate} on an asset is the rate at which the nominal value or purchasing power of the asset increases over time.

    \begin{align*}
    \text{real interes rate} &= \text{nominal interest rate} - \text{inflation} \\
    &= i - \pi
    \end{align*}
\end{definition}

At the time of lending, the nominal interest rate is known, but the real interest rate is unknown because the rate of inflation over the loan period is unknown.

\begin{definition}
    \textbf{Expected real interest rate} is the nominal interest rate minus expected rate of inflation. 
    \[
        r = i - \pi^e
    \]
\end{definition}

\pagebreak





