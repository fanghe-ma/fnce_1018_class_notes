\section{Asset Market, Money and Prices}

\subsection{Money}

\begin{definition}
    (Money): Money refers to assets that are widely used and accepted as payment. Money has 3 functions 
    \begin{itemize}
        \item medium of exchange
        \item store of value
        \item unit of account
    \end{itemize} 
\end{definition}

\begin{definition}
    (M1): The M1 monetary aggregate is the narrowest measure, consisting of
    \begin{itemize}
        \item currency
        \item travelers' check
        \item demand deposits 
        \item other checkable deposits
    \end{itemize} 
\end{definition}
\begin{definition}
    (M2): The M1 monetary aggregate is the narrowest measure, consisting of
    \begin{itemize}
        \item M1
        \item Savings deposits 
        \item Time deposits 
        \item Money market mutual funds
    \end{itemize} 
\end{definition}

\begin{definition}
    (Money supply): Money supply is the amount of money available in an economy. \\

    Central banks can control money supply through \textbf{open market operations}, and 
    \begin{itemize}
        \item increase money supply by \textbf{buying financial assets}
        \item decrease money supply by \textbf{selling financial assets}
    \end{itemize} 
\end{definition}

\subsection{Money Demand}

\begin{definition}
    (Money demand): The demand for money is the quantity of monetary assets that people choose to hold in their portfolios.
    \[
        M^d = PL (Y, i) = PL(Y, r + \pi^e) implies \frac{M^d}{p} = L(Y, r + \pi^e)
    \]
    where 
    \begin{itemize}
        \item $M^d$ is nominal aggregate money demand 
        \item $P$ is the price level 
        \item $Y$ is real income 
        \item $i$ is the nominal interest rate earned by \textbf{alternative, non-monetary assets}
        \item $L$ is some function relating money demand to $Y, r$
        \item $r$ is real interest rate
        \item $\pi^e$ is expected inflation
        \item $M^d / P$ is the real money demand
    \end{itemize} 
\end{definition}

\begin{remark}
    Factors affecting money demand 
    \begin{itemize}
        \item price level 
        \begin{itemize}
            \item higher price level $\implies$ people need more dollars to conduct transactions $\implies$ people hold more money (proportional)
            \item 
        \end{itemize} 
        \item real income
        \begin{itemize}
            \item higher real income $\implies$ more transactions $\implies$ more liquidity needed (less than proportionate)
        \end{itemize} 
        \item interest rates 
        \begin{itemize}
            \item increase in expected return on money $\implies$ more demand for money 
            \item increase in expected return on non-monetary assets $\implies$ less demand for money
        \end{itemize} 
    \end{itemize} 
\end{remark}

\begin{definition}
    \textbf{Velocity of money}: The velocity of money measures how often the money stock turns over each period 
    \[
        V = \frac{PY}{M} = \frac{\text{nominal GDP}}{\text{nominal money stock}}
    \]
\end{definition}

\begin{theorem}
    (Quantity theory of money): The quantity theory of money says that real money demand is proportional to real income 
    \[
        \frac{M^d}{P} = kY
    \]

    Where $k$ is some constant. 
\end{theorem}

\subsection{Asset market equilibrium}
\begin{definition}
    (Asset market equilibrium): The asset market equilibrium condition is when 
    \[
        \frac{M}{P} = L(Y, r + \pi^e)
    \]
\end{definition}
\begin{remark}
    \begin{center}
    \includegraphics[width = 0.5\textwidth]{graphs/fig_7_1.jpg}
    \end{center}
\end{remark}


\begin{remark}
    \[
        P = \frac{M}{L(Y, r + \pi^e)}
    \]
    This implies
    \begin{itemize}
        \item $P$ is proportional to $M$ (price level proportional to nominal money supply)
    \end{itemize} 
\end{remark}

\subsection{Money growth and inflation}
\begin{definition}
    (Inflation): The inflation rate is the growth rate of the price level 
    \[
        \pi = \frac{\triangle P}{P} = \frac{\triangle M}{M} - \frac{\triangle L(Y, r + \pi^e)}{L(Y, r + \pi^e)}
    \]
    At asset market equilibrium, rate of inflation equals growth rate of nominal money supply minus growth rate of real money demand. \\

    This can be expressed as 
    \[
        \pi = \frac{\triangle P}{P} = \frac{\triangle M}{M} - \eta_Y \frac{\triangle Y}{Y} - \eta_i \frac{\triangle i}{i}
    \]
    where 
    \begin{itemize}
        \item $i$: nominal interest rate 
        \item $\eta_i$: interest elasticity of money demand
    \end{itemize} 
    

    Assuming that change in real income is the only source of change in money demand, 
    \[
        \pi = \frac{\triangle P}{P} = \frac{\triangle M}{M} - \eta_{Y} \frac{\triangle Y}{Y}
    \]
    
\end{definition}



\newpage










