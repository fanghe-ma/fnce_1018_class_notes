%%%%%%%%%%%%%%%%%%%%%%%%%%%%%%%%%%%%%%%%%%%%%%%%%%%%%%%%%%%%%%%%%%%%%%
% Source: Dave Richeson (divisbyzero.com), Dickinson College
% Version francaise par Vincent Pantaloni, prof.pantaloni.free.fr
% Traduction, correction et adaptation à la typographie française.
% 
% Une anti-seche en deux pages pour une intro rapide ou un aide mémoire des différentes fonctions. A imprimer en recto verso par exemple.
%
% Feel free to distribute this example, but please keep the referral
% to divisbyzero.com
% 
%%%%%%%%%%%%%%%%%%%%%%%%%%%%%%%%%%%%%%%%%%%%%%%%%%%%%%%%%%%%%%%%%%%%%%
%
%%%%%%%%%%%%%%%%%%%%%%%%%%%%%%%%%%%%%%%%%%%%%%%%%%%%%%%%%%%%%%%%%%%%%%

\documentclass[letterpaper, 9pt,landscape]{extarticle}
\usepackage{amsthm} 
\usepackage{amsmath}
\usepackage{amssymb}
\usepackage{ifthen}
\usepackage{calc}
\usepackage{geometry}
\usepackage{multicol}
\usepackage{graphicx}
\usepackage{xcolor}
\usepackage{hyperref}
\usepackage{environ}
\usepackage{enumitem}
\usepackage{etoolbox}

\ifthenelse{\lengthtest{\paperwidth = 11in}}
  {\geometry{paper=letterpaper, landscape, top=.2in,left=.2in,right=.2in,bottom=.2in}}
  {\ifthenelse{\lengthtest{\paperwidth = 297mm}}
     {\geometry{paper=a4paper, landscape, top=1cm,left=1cm,right=1cm,bottom=1cm}}
     {\geometry{landscape, top=1cm,left=1cm,right=1cm,bottom=1cm}}}

\newif\ifbw
\ifdefined\BW
  \bwtrue
\else
  \bwfalse
\fi

\setcounter{secnumdepth}{0}

\makeatletter
\newtheoremstyle{notitle}%
  {3pt}% space above
  {3pt}% space below
  {}% body font
  {}% indent amount
  {\bfseries}% theorem head font (bold just in case)
  {}% punctuation after theorem head
  {0pt}% space after theorem head
  {}% theorem head spec
\makeatother

\theoremstyle{notitle}
\newtheorem*{definition}{}
\newtheorem*{theorem}{}
\newtheorem*{lemma}{}
\newtheorem*{proposition}{}
\newtheorem*{corollary}{}
\newtheorem*{remark}{}

\newcommand{\maybeCol}[1]{\ifbw black\else #1\fi} 
\newcommand{\maybeBack}[1]{\ifbw white\else #1\fi}

\ifbw
  \hypersetup{colorlinks=false, hidelinks}
\else
  \hypersetup{colorlinks=true, citecolor=blue, linkcolor=blue, urlcolor=blue}
\fi

\pagestyle{empty}

\makeatletter
\renewcommand{\section}{\@startsection{section}{1}{0mm}%
  {-1ex plus -.5ex minus -.2ex}%
  {0.5ex plus .2ex}%
  {\normalfont\large\bfseries}}
\renewcommand{\subsection}{\@startsection{subsection}{2}{0mm}%
  {-1ex plus -.5ex minus -.2ex}%
  {0.5ex plus .2ex}%
  {\normalfont\normalsize\bfseries}}
\renewcommand{\subsubsection}{\@startsection{subsubsection}{3}{0mm}%
  {-1ex plus -.5ex minus -.2ex}%
  {1ex plus .2ex}%
  {\normalfont\small\bfseries}}
\makeatother

\setcounter{secnumdepth}{0}
\setlength{\parindent}{0pt}
\setlength{\parskip}{0pt plus 0.5ex}

\setlist[itemize]{leftmargin=*, itemsep=0pt, topsep=0.25ex}
\setlist[enumerate]{leftmargin=*, itemsep=0pt, topsep=0.25ex}

\newcommand{\CheatSheetSetup}{%
  \raggedright
  \footnotesize
}

\newlength{\CheatColSep}
\setlength{\CheatColSep}{2pt}

\NewEnviron{cheatsheet}[1][5]{%
  \begin{multicols*}{#1}
    \setlength{\premulticols}{1pt}% space above first column
    \setlength{\postmulticols}{1pt}% space below last column
    \setlength{\multicolsep}{1pt}% vertical spacing between multi-page chunks
    \setlength{\columnsep}{\CheatColSep}% column gap
    \BODY
  \end{multicols*}
}

\newcommand{\accentblue}[1]{\textcolor{\maybeCol{blue}}{#1}}
\newcommand{\accentred}[1]{\textcolor{\maybeCol{red}}{#1}}
\newcommand{\accentgreen}[1]{\textcolor{\maybeCol{green!40!black}}{#1}}

\newcommand{\cheatrule}{\par\vspace{2pt}\noindent\color{\maybeCol{black!60}}%
  \rule{\linewidth}{0.3pt}\par\vspace{2pt}}

\newcommand{\cheatgraphic}[1]{\includegraphics[width=\linewidth]{#1}}

\colorlet{CheatRule}{\maybeCol{black!60}}
\colorlet{CheatLink}{\maybeCol{blue}}
\colorlet{CheatBack}{\maybeBack{white}}


\numberwithin{equation}{section}

% \theoremstyle{plain}
% \newtheorem{theorem}{Theorem}[section]
% \newtheorem{lemma}[theorem]{Lemma}
% \newtheorem{proposition}[theorem]{Proposition}
% \newtheorem{corollary}[theorem]{Corollary}
% \newtheorem{claim}[theorem]{Claim}

% \theoremstyle{definition}
% \newtheorem{definition}[theorem]{Definition}
% \newtheorem{assumption}[theorem]{Assumption}
% \newtheorem{example}[theorem]{Example}
% \newtheorem{algorithm}[theorem]{Algorithm}

% \theoremstyle{remark}
% \newtheorem{remark}[theorem]{Remark}
% \newtheorem{note}[theorem]{Note}
% \newtheorem{warning}[theorem]{Warning}

% \newtheorem*{theorem*}{Theorem}
% \newtheorem*{lemma*}{Lemma}
% \newtheorem*{proposition*}{Proposition}
% \newtheorem*{corollary*}{Corollary}
% \newtheorem*{definition*}{Definition}
% \newtheorem*{remark*}{Remark}
% \newtheorem*{example*}{Example}

\renewcommand{\qedsymbol}{$\blacksquare$} 


% \usepackage{fontspec}
% \usepackage[T1]{fontenc}
\usepackage{amssymb,amsmath,amsthm,amsfonts}
\usepackage{multicol,multirow}
\usepackage{calc}
\usepackage{ifthen}
\usepackage{pdfpages}
\usepackage{graphicx}
\usepackage{tikz}
\usepackage{pgfplots}
\usepackage{stmaryrd}
\usepackage{mathtools}
% \usepackage[landscape]{geometry}

\ifthenelse{\lengthtest { \paperwidth = 11in}}
    { \geometry{top=.2in,left=.2in,right=.2in,bottom=.2in} }
	{\ifthenelse{ \lengthtest{ \paperwidth = 297mm}}
		{\geometry{top=1cm,left=1cm,right=1cm,bottom=1cm} }
		{\geometry{top=1cm,left=1cm,right=1cm,bottom=1cm} }
	}
\pagestyle{empty}
\makeatletter
\renewcommand{\section}{\@startsection{section}{1}{0mm}%
                                {-1ex plus -.5ex minus -.2ex}%
                                {0.5ex plus .2ex}%x
                                {\normalfont\large\bfseries}}
\renewcommand{\subsection}{\@startsection{subsection}{2}{0mm}%
                                {-1explus -.5ex minus -.2ex}%
                                {0.5ex plus .2ex}%
                                {\normalfont\normalsize\bfseries}}
\renewcommand{\subsubsection}{\@startsection{subsubsection}{3}{0mm}%
                                {-1ex plus -.5ex minus -.2ex}%
                                {1ex plus .2ex}%
                                {\normalfont\small\bfseries}}
\makeatother
\setcounter{secnumdepth}{0}
\setlength{\parindent}{0pt}
\setlength{\parskip}{0pt plus 0.5ex}
% -----------------------------------------------------------------------

\begin{document}

\raggedright
\footnotesize

\begin{multicols*}{5}
\setlength{\premulticols}{1pt}
\setlength{\postmulticols}{1pt}
\setlength{\multicolsep}{1pt}
\setlength{\columnsep}{2pt}

\begin{definition}
    \textbf{Gross Domestic Product (Product Approach)} is defined as the market value of final goods and services newly produced within a nation during a fixed period of time.
\end{definition}

\textbf{Market value}: prices at which goods and services are sold
\begin{itemize}
    \item (+): Allows adding of production of different goods and services
    \item (-): Some goods or services are not sold in formal markets, denoted \textbf{nonmarket goods and services}.
\end{itemize}

\textbf{Newly produced}: GDP only includes goods and services produced within current period.

\textbf{Final}: Goods and services are final if they are not intermediate. Goods and services are intermediate if they are used up in the production of others \textbf{in the same period they were produced}. \textbf{inventories} are a final good. 

\begin{definition}
    \textbf{Gross National Product} is defined as the market value of final goods and services newly produced \textit{by domestic factors of production} during the current period.
\end{definition}

\begin{definition}
    \textbf{NFP} or \textbf{Net factor payments from abroad} is defined as the income paid to domestic factors of production by the rest of the world minus income paid to foreign factors of production by the domestic economy.
    \[
        \text{GDP} = \text{GNP} - \text{NFP}
    \]
\end{definition}

\begin{definition}
    \textbf{Income-Expenditure Identity}: The \textbf{Gross Domestic Product} (Expenditure Approach) is defined as the total spending on final goods and services produced within a nation during a specified period of time. 
    \[
        Y = C + I + G + NX
    \]
\end{definition}


\textbf{Investment} includes spending on
\begin{itemize}
    \item fixed investment: spending on new capital goods
    \begin{itemize}
        \item business fixed investment
        \item residential investment
    \end{itemize} 
    \item inventory investment: spending on inventory holdings
    \begin{itemize}
        \item including produced goods that are unsold
    \end{itemize} 
\end{itemize} 

\textbf{Government purchases} of goods and services include expenditure by government for a currently produced good or service, foreign or domestic. \\

Note that $G$ excludes
\begin{itemize}
    \item transfers 
    \item interest payment on national debt
\end{itemize} 

Government spending on fixed investments or inventory is counted under $G$, not $I$.

\begin{remark}

\[
\underbrace{
\underbrace{
    \text{National Income}
    + \text{Statistical Discrepancy}
}_{\text{Net National Product}}
+ \text{Depreciation}
}_{\text{Gross National Product}}
+ \text{NFP}
= \text{Gross Domestic Product}
\]
\end{remark}

\begin{definition}
    \textbf{Private disposable income} is the amount of income that the private sector has available to spend.
    \[
        \text{private disposable income} = Y + NFP + TR + INT - T
    \]
\end{definition}

\begin{definition}
    \textbf{Net government income} is taxes paid by private sector, minus payments from government to the private sector 
    \[
        \text{net government income} = T - TR - INT
    \]
\end{definition}

\begin{definition}
    \textbf{Private saving} is defined as the difference of private disposable income and consumption 
    \begin{align*}
        S_{pvt} = (Y + NFP - T + TR + INT) - C
    \end{align*}
\end{definition}


\begin{remark}
    Note that \textbf{investments} is not subtracted from private disposable income even though it constitutes private spending, because it \textit{does not satisfy current needs}.
\end{remark}

\begin{definition}
    \textbf{Government saving} is defined as net government income, less goernment purchases of goods and services 
    \begin{align*}
        S_{govt} = (T - TR - INT) - G
    \end{align*}
\end{definition}

\begin{remark}
    \textbf{Government budget surplus} is defined as government receipts less outlays 
    \begin{align*}
        \text{budget surplus} &= T - (G + TR + INT) \\
        &= S_{govt}
    \end{align*}
\end{remark}

\begin{theorem}
    \textbf{Uses-of-saving identity} The economy's private saving is used in 3 ways 
    \[
        S_{pvt} = I + (- S_{govt}) + CA
    \]
\end{theorem}

\begin{definition}
    \textbf{National Wealth} is the total wealth of residents in a country, consisting of 
    \begin{itemize}
        \item domestic physical assets, i.e. capital goods and land
        \item net foreign assets, comprising of 
        \begin{itemize}
            \item (+) foreign physical and financial assets 
            \item (-) foreign physical and financial liabilities 
        \end{itemize} 
    \end{itemize} 
\end{definition}


\begin{definition}
    The \textbf{GDP deflator} is a price index that measures the overall level of prices of goods and services included in GDP. 
    \[
        \text{GDP deflator} = 100 \times \frac{ \text{nominal GDP}}{ \text{real GDP}}
    \]
\end{definition}


\begin{definition}
    The \textbf{Consumer Price Index}, or \textbf{CPI}, measures the prices of consumer goods, calculated as 100 times the current cost of a specific baset of consumer items divided by the cost of the same basket in the reference base period.
\end{definition}

\begin{remark}
    The CPI tends to overstate increases in cost of living because of
    \begin{itemize}
        \item improvements in quality of goods and services
        \item substition bias
        \begin{itemize}
            \item CPI doesn't account for substitution away from the specified basket of goods
        \end{itemize} 
    \end{itemize} 
\end{remark}

\begin{definition}
    \textbf{Inflation rate} for a given period is defined as the percentage change in the price index in the same period.
    \[
        \pi_{t + 1} = \frac{P_{t + 1} - P_t}{P_t} = \frac{\triangle P_{t+1}}{P_t}
    \]
\end{definition}

\begin{definition}
    The \textbf{real interest rate} on an asset is the rate at which the real value or purchasing power of the asset increases over time. \\


    The \textbf{nominal interest rate} on an asset is the rate at which the nominal value or purchasing power of the asset increases over time.

    \begin{align*}
    \text{real interes rate} &= \text{nominal interest rate} - \text{inflation} \\
    &= i - \pi
    \end{align*}
\end{definition}

\begin{definition}
    \textbf{Expected real interest rate} is the nominal interest rate minus expected rate of inflation. 
    \[
        r = i - \pi^e
    \]
\end{definition}

\begin{definition}
    The \textbf{Marginal product of capital} is the increase in output produced that results from a one-unit increase in capital stock. 
    \[
        MPK = \frac{\triangle Y}{\triangle K}
    \]
\end{definition}

\begin{definition}
    The \textbf{Marginal product of labor} is the increase in output produced that results from a one-unit increase in labor. 
    \[
        MPN = \frac{\triangle Y}{\triangle N}
    \]
\end{definition}

\begin{remark}
    The $MPK$ and $MPN$
    \begin{itemize}
        \item are positive 
        \item are decreasing in $K$ / $N$, due to diminishing marginal product
    \end{itemize} 
\end{remark}

\begin{remark}
Given a decrease in $A$, 
    \begin{itemize}
        \item the marginal product decreases for every value of $N$
        \item the amount of output decreases for every value of $N$
    \end{itemize} 
    \centering
    \begin{tikzpicture}
        \begin{axis}[
                scale = 0.5,
                xmin = 0, xmax = 10,
                ymin = 0, ymax = 10,
                axis lines* = left, 
                xtick = {0}, ytick = \empty,
                clip = false,
            ]
            % Y = AK^{0.3}N^{0.7}, with A=3, K=2
            \addplot[
                domain = 0:10,
                restrict y to domain = 0:10,
                samples = 800,
                color = black,
            ]{3*x^0.3*2^0.7};
            \addplot[
                domain = 0:10,
                restrict y to domain = 0:10,
                samples = 800,
                color = red,
            ]{1.5*x^0.3*2^0.7};
            \node[right] at (current axis.right of origin) {Labor, $N$};
            \node[above] at (current axis.above origin) {Output, $Y$};
        \end{axis}
    \end{tikzpicture}
    \textbf{Fig - Decrease in $A$, holding $K$ fixed}
    Note that 
    \begin{align*}
        &A_2 < A_1 \\
        \implies&  Y_2 < Y_1 \\
        \implies&  MPN_2 <  MPN_1
    \end{align*}
\end{remark}

\begin{definition}
    \textbf{Marginal revenue product of labor} is the benefit of employing an additioinal worker in terms of the extra revenue produced. 
    \[
        MRPN = P \times MPN
    \]
\end{definition}

The quantity of labor demanded is 
\begin{itemize}
    \item in nominal terms: equal to the $MRPN$
    \item in real terms: equal to $MPN$
\end{itemize} 

\begin{definition}
    \text{Real wage} refers to the wage measured in terms of units of output.
    \[
        w = \frac{W}{P}
    \]
\end{definition}

\begin{remark}
    Note that firms will want to increase employment under the following condition, all 4 statements are equivalent
    \begin{align*}
        & MPN > w  \\
        \iff & MPN > \frac{W}{P}  \\
        \iff & P \times MPN > W  \\
        \iff & MPRN > W 
    \end{align*}

    Vice versa for condition under which the firm will want to decrease employment
\end{remark}

\begin{remark}
Factors that affect labor demand must change the amount of labor that firms want to employ \textit{at any given level of the real wage}.\\

The labor demand increases in response to 
\begin{itemize}
    \item $A \uparrow$, productivity improvements / positive supply shock
    \item $K \uparrow$, increase in capital supply
\end{itemize} 
\end{remark}

\begin{definition}
    The \textbf{substitution effect} refers to an increase in the opportunity cost of leisure causing workers to substitute away from leisure towards work.
\end{definition}

\begin{remark}
    \textbf{Pure substitution effect}: one-day rise in real wage, $NS \uparrow$.  \\
\end{remark}

\begin{definition}
    The \textbf{income effect} refers to workers being better off and hence working less.
\end{definition}
\begin{remark}
    \textbf{Pure income effect}: changes in $Z$, e.g. winning the lottery, or higher expected future real wages, $NS \downarrow$ \\
\end{remark}

\begin{remark} 
    \textbf{Income} and \textbf{substition} effect work in opposite directions on labor supply.  \\

    An increase in real wages 
    \begin{itemize}
        \item raises the marginal benefit of work, increases labor supply, by \textbf{substitution effect}
        \item increases workers' wealth, decreases labor supply, by \textbf{income effect}
    \end{itemize}   
\end{remark}


\begin{remark}
The labor supply shifts left in response to 
\begin{itemize}
    \item increases in weath, $NS \downarrow$
    \item increases in expected future real wage, $NS \downarrow$
    \item decrease in working age population
    \item decrease in participation rate
\end{itemize} 
\end{remark}

\begin{definition}
    \textbf{Full-employment level of employment}, $ \bar{N} $ is defined as the equilibrium level of employment. The corresponding market-clearing real wage is $ \bar{w} $.
\end{definition}

\begin{definition}
    \textbf{Full-employment output}, $\bar{Y}$, also called \textbf{potential output}, is the level of output that firms in the economy supply when wages and prices have fully adjusted. 

    $\bar{Y} $ is achieved when aggregate employment reaches its full-employment level, $\bar{N}$
    \[
        \bar{Y}  = AF \left( K, \bar{N}  \right) 
    \]
\end{definition}
\begin{remark}
    A decrease in $A$ reduces $\bar{Y} $ in two ways
    \begin{itemize}
        \item  $A \downarrow \rightarrow \bar{Y} \downarrow$ directly 
        \item  $A \downarrow \rightarrow MPN \downarrow \rightarrow ND \downarrow \rightarrow \bar{N} \downarrow \rightarrow \bar{Y} \downarrow $ 
    \end{itemize} 
\end{remark}

\begin{remark}
    Classification of individuals:
    \begin{itemize}
        \item $E$, employed, if person worked full-time or part-time during the past week
        \item $U$, unemployed, if person didn't work during the past week but looked for work in the past four weeks
        \item $NLF$, not in labor force, if the person didnt work and didn't look for work in the past 4 weeks
        \begin{itemize}
            \item discouraged workers, people who become discouraged and move from $U$ to $NLF$
        \end{itemize} 
    \end{itemize} 
    \begin{align*}
        \text{labor force} &= LF =  E + U \\
        \text{adult population} &= LF + NLF \\
        \text{participation rate} &= \frac{LF}{LF + NLF} \\
        \text{employment ratio} &= \frac{E}{LF + NLF}
    \end{align*}
\end{remark}

\begin{remark}
    Sources of unemployment 
    \begin{itemize}
        \item \textbf{frictional unemployment}: arises as workers search for suitable jobs and firms search for suitable workers
        \item \textbf{structural unemployment}: long-term and chronic unemployment that exists even when the economy is not in a recession
        \begin{itemize}
            \item unskilled, low skilled workers
            \item rellocation of labor from shrinking industries / depressed regions
        \end{itemize} 
    \end{itemize} 
\end{remark}

\begin{definition}
    The \textbf{natural rate of unemployment}, $ \bar{u} $ is the rate of unemployment that prevails when output and employment are at the full-employment level. \\

    The difference betwen actual unemployment and natural unemployment is \textbf{cyclical unemployment} 
    \[
    \text{cyclical unemployment}= u - \bar{u} 
    \]
\end{definition}

\begin{theorem}
    \textbf{Okun's Law} states that the gap between full-employment output and actual output increases by 2 percent for each percent increase in unemployment 
    \[
    \frac{\bar{Y} - Y }{\bar{Y} } = 2 \left( u  - \bar{u}  \right) 
    \]

    Alternatively, the percentage change in real output is roughly 3 percent minus two times the change in unemployment
    \[
    \frac{\triangle Y}{Y} = \frac{\triangle \bar{Y} }{ \bar{Y} }  - 2 \triangle u
    \]
\end{theorem}


\begin{definition}
    An individual's \textbf{Present Value of Lifetime Resources, $PVLR$} is defined as 
    \[
        PVLR = a + y + \frac{y^f}{1 + r}
    \]
\end{definition}

\begin{definition}
    An individual's \textbf{Present Value of Lifetime Consumption, $PVLC$} is defined as 
    \[
        PVLC = c + \frac{c^f}{1 + r}
    \]
\end{definition}



\begin{definition}
    An individual's \textbf{budget constraint} is given by
    \begin{align*}
        PVLC &= PVLR \\
        c + \frac{c^f}{1 + r} &= (a + y) + \frac{y^f}{1+r} \\
        c^f &= (a + y - c) (1 + r) + y^f \\
        &= \underbrace{(a + y)(1 + r)+ y^f}_{intercept}   \underbrace{ - (1+r)}_{slope} c 
    \end{align*}
\end{definition}

\begin{remark}
    We can classify individuals as lending or borrowing 
    \begin{itemize}
        \item lending, if $c < a + y \iff a + y - c > 0$
        \item borrowing if $c > a + y \iff a + y - c < 0$
    \end{itemize} 
\end{remark}


\begin{remark}
    We can classify individuals as saving or dissaving
    \begin{itemize}
        \item saving, if $y > c$
        \item dissaving, if $y < c$
        \begin{itemize}
            \item borrowing, if $c > y + a \iff a + y - c < 0$
        \end{itemize} 
    \end{itemize} 

    \textbf{Dissaving} $\neq$ \textbf{Borrowing}.
\end{remark}


\begin{remark}
    \textbf{Slope of indifference curve}
\begin{center}
\includegraphics[width = 0.5\columnwidth]{../graphs/fig_4_5.jpg}
\end{center}
    \begin{itemize}
        \item $U_1$: \textbf{present-oriented}, steeper, values consumption today, require a lot of consumption to give up a unit of consumption today
        \item $U_2$: \textbf{future-oriented}, flatter, require a lot of consumption today to give up a unit of consumption tomorrow
    \end{itemize} 
\end{remark}


\begin{remark}
    Income effect occurs when 
    \begin{itemize}
        \item $a \uparrow$
        \item $y \uparrow$
        \item $y^f \uparrow$
    \end{itemize} 
    As a result
    \begin{itemize}
        \item $PVLR \uparrow$
        \item $r$ unchanged
    \end{itemize} 

    \textbf{Income effect} operates through $PVLR$ with unchanged $r$.
\begin{center}
\includegraphics[width = 0.5\columnwidth]{../graphs/fig_4_12.jpg}
\end{center}
\end{remark}

\begin{theorem}
    The \textbf{Riccardian Equivalence Proposition} states that tax cuts do not affect desired consumption or national saving because in the long run, because all government purchases must be paid for by taxes. 
\end{theorem}

\begin{remark}
    \textbf{Riccardian Assumptions}
    \begin{enumerate}
        \item Assuming REP does not hold 
        \begin{itemize}
            \item People spend some and save some, i.e. \textit{consumption smoothing}
            \[
            T \downarrow \implies c \uparrow, s \uparrow
            \]
        \end{itemize} 
        \item Assuming REP holds and \textbf{there are borrowing constraints}
        \begin{itemize}
            \item Borrowers facing constraints increase consumption, 
            \[
            T \downarrow \implies c \uparrow, s \downarrow
            \]
        \end{itemize}
        \item Assuming REP holds and there are no borrowing constraints 
        \[
        T \downarrow \implies c, s \text{ constant }
        \]
    \end{enumerate}
\end{remark}


\begin{remark}
    One time tax rebate, assuming REP with borrowing constraints
    
    \begin{center}
    \includegraphics[width = 0.4\columnwidth]{../graphs/fig_4_6.jpg}
    \end{center}

    If 
    \begin{itemize}
        \item $a + \tilde{y} < c^*$ increase $c$ by the full amount of the tax rebate
        \item $a + \tilde{y} > c^*$ increase $c$ only up to $c^*$
    \end{itemize} 
\end{remark}


\begin{remark}
    \textbf{Effect of an increase in interest rate on budget constraint}  
    \begin{itemize}
        \item $PVLR \downarrow$: present value of future income decreases
        \item vertical intecept $\uparrow$: future value of present income and assets increases 
        \item no-borrowing no-lending point remains the same 
        \item $c^*$: depends
    \end{itemize} 

    \begin{center}
    \includegraphics[width = 0.5\columnwidth]{../graphs/fig_4_1.jpg}
    \end{center}
\end{remark}

\begin{remark}
    \textbf{Effect of an increase in interest rate} \\
    \begin{minipage}{0.48\columnwidth}
        Borrowers unequivocally consume less.
        \begin{itemize}
            \item Substitution effect: $r \uparrow \implies s \uparrow c \downarrow$
            \item Income effect: $r \uparrow \implies s \uparrow c \downarrow$
        \end{itemize} 
    \begin{center}
    \includegraphics[width = 0.8\columnwidth]{../graphs/fig_4_2.jpg}
    \end{center}
    \end{minipage}
    \hfill
    \begin{minipage}{0.48\columnwidth}
        Lenders consumption uncertain
        \begin{itemize}
            \item Substitution effect: $r \uparrow \implies s \uparrow c \downarrow$
            \item Income effect: $r \uparrow \implies s \downarrow c \uparrow$
        \end{itemize} 
    \begin{center}
    \includegraphics[width = 0.8\columnwidth]{../graphs/fig_4_3.jpg}
    \end{center}
    \end{minipage}
\end{remark}

\begin{remark}
    \textbf{Summary of factors affecting consumption}. \\
    \begin{center}
    \begin{tabular}{c | c | c}
    \textbf{Change} & $\Delta C$ & $\Delta S$ \\ \hline
    $y \uparrow$     & $c \uparrow$     & $s \uparrow$     \\
    $a \uparrow$     & $c \uparrow$     & $s \downarrow$   \\
    $y^{f} \uparrow$ & $c \uparrow$     & $s \downarrow$   \\
    $r \uparrow$     & $c \downarrow$   & $s \uparrow$     \\
    \end{tabular}
    \end{center}

\end{remark}

\begin{definition}
    A firm's \textbf{desired capital stock} is the profit-maximizing amount of capital for the firm.
\end{definition}

\begin{remark}
    The profit-maximizing level of capital is achieved when the expected future marginal benefit, \textit{expected future marignal product of capital}, $MPK^f$ is equal to the expected future marginal cost, \textit{user cost of capital}.
\end{remark}

\begin{definition}
    The \textbf{user cost of capital} is the expected real cost of a unit of capital for a specific period of time.  
    \[
        uc = (r + d) p_K
    \]
\end{definition}

\begin{remark}
    \textbf{Desired Capital Stock}
    \begin{center}
    \includegraphics[width = 0.5\columnwidth]{../graphs/fig_4_7.jpg}
    \end{center}
\end{remark}

\begin{definition}
    The \textbf{tax-adjusted user cost of capital} is the user cost of capital divided by $1 + \tau$ where $\tau$ is the tax rate on firm revenues. \\
    \[
        \frac{uc}{1 - \tau} = \frac{(r + d)p_K}{1 - \tau}
    \]
\end{definition}

\begin{definition}
    \textbf{Gross investment} is defined as the total purchase or construction of new capital goods. 
\end{definition}

\begin{definition}
    \textbf{Net investment} is defined as the difference between gross investment and depreciation.
    \[
    \underbrace{K_{t + 1} - K_t}_{\text{net investment}} = \underbrace{I_t}_{\text{gross investment}} - \underbrace{dK_t}_{ \text{ depreciation}}
    \]
\end{definition}

\begin{remark}
    Summary of factors affecting goods-market equilibrium  \\

    \begin{center}
    \begin{tabular}{l l c c}
    	\textbf{Change} & $\Delta C$ & $\Delta S$ \\\hline
    $A \downarrow$ & $\downarrow$ & $\downarrow$ \\
    $G \uparrow$ & $\shortdownarrow$ a little & $\downarrow$ \\
    Wealth $\uparrow$ & $\uparrow$ & $\downarrow$ \\
    $T \downarrow$ & $\uparrow$ followed by  $\shortdownarrow$ a little & $\uparrow$ \\
    $\tau \downarrow$ & $\shortdownarrow$ a little & $\shortuparrow$ a little \\
    \end{tabular}
    \end{center}
\end{remark}




\end{multicols*}
\end{document}
